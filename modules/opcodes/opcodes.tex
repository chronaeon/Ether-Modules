\documentclass[10pt,a4paper,oneside]{scrartcl}
\usepackage[latin1]{inputenc}
\usepackage{amsmath}
\usepackage{amsfonts}
\usepackage{amssymb}
\usepackage{makeidx}
\usepackage{graphicx}
\usepackage{booktabs}
\usepackage[
	   style=ieee
	   ]
	   {biblatex}
\usepackage{mathtools}
\usepackage{longtable}
\usepackage{array}
\newcolumntype{P}[1]{>{\centering\arraybackslash}p{#1}}
\author{}
\title{Opcodes}
\date{}
\addbibresource{~/modules/References.bib}
\begin{document}
\maketitle
\paragraph{Notation}: \texttt{opcode}
\paragraph{Description}: The EVM has several opcodes which function as an assembly language:


		\subsubsection{EVM Code}
				\paragraph{}The bytecode that the EVM can natively execute. Used to explicitly specify the meaning of a message to an account.
				\paragraph{Notation}: \texttt{contract}
				\paragraph{Description}: A piece of EVM Code that may be associated with an Account or an Autonomous Object. 

			\subsubsection{Opcodes/EVM Assembly}
		The human readable version of EVM code. But what exactly are these computer instructions that can be executed with the same level of veracity and certainty as Bitcoin transactions? How do they come about, what makes them up, how are they kept in order, and what makes them execute? The first part of answering this question is understanding opcodes. In traditional machine architectures, you may not be introduced to working with processor-level assembly instructions for some time. In Ethereum however, they are essential to understanding the protocol because they are the most minute and subtle (yet HUGELY important) things going on in the Ethereum Blockchain at any moment, and they are the real "currency," that Ethereum trades in. I'll explain what I mean by that in a minute. First, let's go over a few Opcodes:\footnote{A full list of Opcodes is in Appendix A} \\


				\paragraph{}The STOP Opcode is used in order to stop a computation once it has completed, or to halt a computation if it has run out of gas. The ADD, MUL, SUB, and DIV operations are addition, multiplication, subtraction and division operations. The In/Out columns refer to inputs (to \texttt{potential\_state}), the state which decides every new \texttt{actual\_state}. 


	\section{Opcodes}

		\subsection{0x10's: Comparisons and Bitwise Logic Operations}
			\begin{longtable}{|cP{2.8cm}cccp{6cm}|}
			\hline
			\textbf{Data} & \textbf{Opcode} & \textbf{Gas} & \textbf{Input} & \textbf{Output} & \textbf{Description} \\
			\hline
			0x00 & STOP & 0 & 0 & 0 & Halts execution. \\
			0x01 & ADD & 3 & 2 & 1 & Addition operation. \\
			0x02 & MUL & 5 & 2 & 1 & Multiplication operation. \\
			0x03 & SUB & 3 & 2 & 1 & Subtraction operation. \\
			0x04 & DIV & 5 & 2 & 1 & Integer division operation. \\
			0x05 & SDIV & 5 & 2 & 1 & Signed integer division operation (truncated.)\\
			0x06 & MOD & 5 & 2 & 1 & Modulo remainder operation. \\
			0x07 & SMOD & 5 & 2 & 1 & Signed modulo remainder operation. \\
			0x08 & ADDMOD & 8 & 3 & 1 & Modulo addition operation. \\
			0x09 & MULMOD & 8 & 3 & 1 & Modulo multiplication operation. \\
			0x0a & EXP & 10 & 2 & 1 & Exponential operation. \\
			0x0b & SIGNEXTEND & 5  & 2 & 1 & Extend the length of two's complementary signed integer. \\
			0x10 & LT & 3 & 2 & 1 & Less-than comparison. \\
			0x11 & GT & 3 & 2 & 1 & Greater-than comparison. \\
			0x12 & SLT & 3 & 2 & 1 & Signed less-than comparison. \\
			0x13 & SGT & 3 & 2 & 1 & Signed greater-than comparison. \\
			0x14 & EQ & 3 & 2 & 1 & Equality comparison. \\
			0x15 & ISZERO & 3 & 1 & 1 & Simple not operator. \\
			0x16 & AND & 3 & 2 & 1 & Bitwise \textsc{and} operation. \\
			0x17 & OR & 3 & 2 & 1 & Bitwise \textsc{or} operation. \\
			0x18 & XOR & 3 & 2 & 1 & Bitwise \textsc{xor} operation. \\
			0x19 & NOT & 3 & 1 & 1 & Bitwise \textsc{not} operation. \\
			0x1a & BYTE & 3 & 2 & 1 & Retrieve single byte from word. \\
			\hline
			\end{longtable}

	        \subsection{0x20's: SHA3}
			\begin{longtable}{|cP{2.8cm}cccp{6cm}|}
			\hline
		        \textbf{Data} & \textbf{Opcode} & \textbf{Gas}  & \textbf{Input}  & \textbf{Output} & \textbf{Description} \\
			\hline
			0x20 & SHA3 & 30 & 2 & 1 & Compute a Keccak-256 hash. \\
			\hline
			\end{longtable}

        	\subsection{0x30's: Environmental Information}
			\begin{longtable}{|cP{2.8cm}cccp{6cm}|}
			\hline
	        	\textbf{Data} & \textbf{Opcode} & \textbf{Gas}  & \textbf{Input}  & \textbf{Output} & \textbf{Description} \\
			\hline
			0x30 & ADDRESS & 2 & 0 & 1 & Get the address of the currently executing account. \\
			0x31 & BALANCE & 400 & 1 & 1 & Get the balance of the given account. \\
			0x32 & ORIGIN & 2 & 0 & 1 & Get execution origination address. This is always the original sender of a transaction, never a contract account. \\
			0x33 & CALLER & 2 & 0 & 1 & Get caller address. This is the address of the account that is directly responsible for this execution. \\
			0x34 & CALLVALUE & 2 & 0 & 1 & Get deposited value by the instruction/transaction responsible for this execution. \\
			0x35 & CALLDATALOAD & 3 & 1 & 1 & Get input data of the current environment. \\
			0x36 & CALLDATASIZE & 2 & 0 & 1 & Get size of input data in current environment. This refers to the optional data field that can be passed with a message call instruction or transaction.\\
			0x37 & CALLDATACOPY & 3 & 3 & 0 & Copy input data in the current environment to memory. This refers to the optional data field passed with the message call instruction or transaction. \\
			0x38 & CODESIZE & 2 & 0 & 1 & Get size of code running in the current environment. \\
			0x39 & CODECOPY & 3 & 3 & 0 & Copy the code running in the current environment to memory. \\
			0x3a & GASPRICE & 2 & 0 & 1 & Get the price of gas in the current environment. This is the gas price specified by the originating transaction. \\
			0x3b & EXTCODESIZE & 700 & 1 & 1 & Get the size of an account's code. \\
			0x3c & EXTCODECOPY & 700 & 4 & 0 & Copy an account's code to memory. \\
			0x3d & RETURNDATASIZE & 2 & 0 & 1 & \\
			0x3e & RETURNDATACOPY & 3 & 3 & 0 & \\
			\hline
			\end{longtable}

	        \subsection{0x40's: Block Data}
			\begin{longtable}{|cP{2.8cm}cccp{6cm}|}
			\hline
			\textbf{Data} & \textbf{Opcode} & \textbf{Gas}  & \textbf{Input}  & \textbf{Output} & \textbf{Description} \\
			\hline
			0x40 & BLOCKHASH & 20 & 1 & 1 & Get the hash of one of the 256 most recent blocks. \footnote{A value of 0 is left on the stack if the block number is more than $256$ in number behind the current one, or if it is a number greater than the current one.} \\ 
			0x41 & COINBASE & 2 & 0 & 1 & Look up a block's beneficiary address by its hash.\\
			0x42 & TIMESTAMP & 2 & 0 & 1 & Look up a block's timestamp by its hash.\\
			0x43 & NUMBER & 2  & 0 & 1 & Look up a block's number by its hash. \\
			0x44 & DIFFICULTY & 2 & 0 & 1 & Look up a block's difficulty by its hash. \\
			0x45 & GASLIMIT & 2  & 0 & 1 & Look up a block's gas limit by its hash. \\
			\hline
			\end{longtable}

		\subsection{0x50's: Stack, memory, storage, and flow operations.}
			\begin{longtable}{|cP{2.8cm}cccp{6cm}|}
			\hline
			\textbf{Data} & \textbf{Opcode} & \textbf{Gas}  & \textbf{Input}  & \textbf{Output} & \textbf{Description} \\
			\hline
			0x50 & POP & 2 & 1 & 0 & Removes an item from the stack. \\
			0x51 & MLOAD & 3 & 1 & 1 & Load a word from memory. \\
			0x52 & MSTORE & 3 & 2 & 0 & Save a word to memory. \\
			0x53 & MSTORE8 & 3 & 2 & 0 & Save a byte to memory. \\
			0x54 & SLOAD & 200 & 1 & 1 & Load a word from storage. \\
			0x55 & SSTORE & 0 & 2 & 0 & Save a word to storage.\\
			0x56 & JUMP & 8 & 1 & 0 & Alter the program counter. \\
			0x57 & JUMPI & 10 & 2 & 0 & Conditionally alter the program counter. \\
			0x58 & PC & 2 & 0 & 1 & Look up the value of the program counter prior to the increment resulting from this instruction. \\
			0x59 & MSIZE & 2 & 0 & 1 & Get the size of active memory in bytes. \\
			0x5a & GAS & 2 & 0 & 1 & Get the amount of available gas, including the corresponding reduction for the cost of this instruction. \\
			0x5b & JUMPDEST & 1 & 0 & 0 & Mark a valid destination for jumps. \footnote{This operation has no effect on the \texttt{machine\_state during execution.}} \\
			\hline
			\end{longtable}

		\subsection{0x60-70's: Push Operations}
			\begin{longtable}{|cP{2.8cm}cccp{6cm}|}
        		\hline
        		\textbf{Data} & \textbf{Opcode} & \textbf{Gas}  & \textbf{Input}  & \textbf{Output} & \textbf{Description} \\
        		\hline
			0x60 & PUSH1 & - & 0 & 1 & Place a 1-byte item on the stack. \\
			0x61 & PUSH2 & - & 0 & 1 & Place a 2-byte item on the stack. \\
			0x62 & PUSH3 & - & 0 & 1 & Place a 3-byte item on the stack. \\
			0x63 & PUSH4 & - & 0 & 1 & Place a 4-byte item on the stack. \\
			0x64 & PUSH5 & - & 0 & 1 & Place a 5-byte item on the stack. \\
			0x65 & PUSH6 & - & 0 & 1 & Place a 6-byte item on the stack. \\
			0x66 & PUSH7 & - & 0 & 1 & Place a 7-byte item on the stack. \\
			0x67 & PUSH8 & - & 0 & 1 & Place a 8-byte item on the stack. \\
			0x68 & PUSH9 & - & 0 & 1 & Place a 9-byte item on the stack. \\
			0x69 & PUSH10 & - & 0 & 1 & Place a 10-byte item on the stack. \\
			0x6a & PUSH11 & - & 0 & 1 & Place a 11-byte item on the stack. \\
			0x6b & PUSH12 & - & 0 & 1 & Place a 12-byte item on the stack. \\
			0x6c & PUSH13 & - & 0 & 1 & Place a 13-byte item on the stack. \\
			0x6d & PUSH14 & - & 0 & 1 & Place a 14-byte item on the stack. \\
			0x6e & PUSH15 & - & 0 & 1 & Place a 15-byte item on the stack. \\
			0x6f & PUSH16 & - & 0 & 1 & Place a 16-byte item on the stack. \\
			0x70 & PUSH17 & - & 0 & 1 & Place a 17-byte item on the stack. \\
			0x71 & PUSH18 & - & 0 & 1 & Place a 18-byte item on the stack. \\
			0x72 & PUSH19 & - & 0 & 1 & Place a 19-byte item on the stack. \\
			0x73 & PUSH20 & - & 0 & 1 & Place a 20-byte item on the stack. \\
			0x74 & PUSH21 & - & 0 & 1 & Place a 21-byte item on the stack. \\
			0x75 & PUSH22 & - & 0 & 1 & Place a 22-byte item on the stack. \\
			0x76 & PUSH23 & - & 0 & 1 & Place a 23-byte item on the stack. \\
			0x77 & PUSH24 & - & 0 & 1 & Place a 24-byte item on the stack. \\
			0x78 & PUSH25 & - & 0 & 1 & Place a 25-byte item on the stack. \\
			0x79 & PUSH26 & - & 0 & 1 & Place a 26-byte item on the stack. \\
			0x7a & PUSH27 & - & 0 & 1 & Place a 27-byte item on the stack. \\
			0x7b & PUSH28 & - & 0 & 1 & Place a 28-byte item on the stack. \\
			0x7c & PUSH29 & - & 0 & 1 & Place a 29-byte item on the stack. \\
			0x7d & PUSH30 & - & 0 & 1 & Place a 30-byte item on the stack. \\
			0x7e & PUSH31 & - & 0 & 1 & Place a 31-byte item on the stack. \\
			0x7f & PUSH32 & - & 0 & 1 & Place a 32-byte item on the stack. \\			
			\hline
			\end{longtable}

		\subsection{0x80's: Duplication Operations}
			\begin{longtable}{|cP{2.8cm}cccp{6cm}|}
		        \hline
		        \textbf{Data} & \textbf{Opcode} & \textbf{Gas}  & \textbf{Input}  & \textbf{Output} & \textbf{Description} \\
		        \hline
			0x80 & DUP1 & - & 1 & 2 & Duplicate the 1st item in the stack. \\
			0x81 & DUP2 & - & 2 & 3 & Duplicate the 2nd item in the stack. \\
			0x82 & DUP3 & - & 3 & 4 & Duplicate the 3rd item in the stack. \\
			0x83 & DUP4 & - & 4 & 5 & Duplicate the 4th item in the stack. \\
			0x84 & DUP5 & - & 5 & 6 & Duplicate the 5th item in the stack. \\
			0x85 & DUP6 & - & 6 & 7 & Duplicate the 6th item in the stack. \\
			0x86 & DUP7 & - & 7 & 8 & Duplicate the 7th item in the stack. \\
			0x87 & DUP8 & - & 8 & 9 & Duplicate the 8th item in the stack. \\
			0x88 & DUP9 & - & 9 & 10 & Duplicate the 9th item in the stack. \\
			0x89 & DUP10 & - & 10 & 11 & Duplicate the 10th item in the stack. \\
			0x8a & DUP11 & - & 11 & 12 & Duplicate the 11th item in the stack. \\
			0x8b & DUP12 & - & 12 & 13 & Duplicate the 12th item in the stack. \\
			0x8c & DUP13 & - & 13 & 14 & Duplicate the 13th item in the stack. \\
			0x8d & DUP14 & - & 14 & 15 & Duplicate the 14th item in the stack. \\
			0x8e & DUP15 & - & 15 & 16 & Duplicate the 15th item in the stack. \\
			0x8f & DUP16 & - & 16 & 17 & Duplicate the 16th item in the stack. \\		
			\hline
			\end{longtable}
        
	
		\subsection{0x90's: Swap Operations}
			\begin{longtable}{|cP{2.8cm}cccp{6cm}|}
		        \hline  
		        \textbf{Data} & \textbf{Opcode} & \textbf{Gas}  & \textbf{Input}  & \textbf{Output} & \textbf{Description} \\
		        \hline  
			0x90 & SWAP1 & - & 2 & 2 & Exchange the 1st and 2nd stack items. \\
			0x91 & SWAP2 & - & 3 & 3 & Exchange the 1st and 3rd stack items. \\
			0x92 & SWAP3 & - & 4 & 4 & Exchange the 1st and 4th stack items. \\
			0x93 & SWAP4 & - & 5 & 5 & Exchange the 1st and 5th stack items. \\
			0x94 & SWAP5 & - & 6 & 6 & Exchange the 1st and 6th stack items. \\
			0x95 & SWAP6 & - & 7 & 7 & Exchange the 1st and 7th stack items. \\
			0x96 & SWAP7 & - & 8 & 8 & Exchange the 1st and 8th stack items. \\
			0x97 & SWAP8 & - & 9 & 9 & Exchange the 1st and 9th stack items. \\
			0x98 & SWAP9 & - & 10 & 10 & Exchange the 1st and 10th stack items. \\
			0x99 & SWAP10 & - & 11 & 11 & Exchange the 1st and 11th stack items. \\
			0x9a & SWAP11 & - & 12 & 12 & Exchange the 1st and 12th stack items. \\
			0x9b & SWAP12 & - & 13 & 13 & Exchange the 1st and 13th stack items. \\
			0x9c & SWAP13 & - & 14 & 14 & Exchange the 1st and 14th stack items. \\
			0x9d & SWAP14 & - & 15 & 15 & Exchange the 1st and 15th stack items. \\
			0x9e & SWAP15 & - & 16 & 16 & Exchange the 1st and 16th stack items. \\
			0x9f & SWAP16 & - & 17 & 17 & Exchange the 1st and 17th stack items. \\	
			\hline
			\end{longtable}

		\subsection{0xa0's: Logging Operations}
			\begin{longtable}{|cP{2.8cm}cccp{6cm}|}
		        \hline  
		        \textbf{Data} & \textbf{Opcode} & \textbf{Gas}  & \textbf{Input}  & \textbf{Output} & \textbf{Description} \\
		        \hline  
			0xa0 & LOG0 & 375 & 2 & 0 & Append log record with 0 topics. \\
			0xa1 & LOG1 & 750 & 3 & 0 & Append log record with 1 topic. \\
			0xa2 & LOG2 & 1125 & 4 & 0 & Append log record with 2 topic. \\
			0xa3 & LOG3 & 1500 & 5 & 0 & Append log record with 3 topic. \\
			0xa4 & LOG4 & 1875 & 6 & 0 & Append log record with 4 topic. \\
			\hline
			\end{longtable}

	        \subsection{0xf0's: System Operations}
			\begin{longtable}{|cP{2.8cm}cccp{6cm}|}
		        \hline  
	        \textbf{Data} & \textbf{Opcode} & \textbf{Gas}  & \textbf{Input}  & \textbf{Output} & \textbf{Description} \\
			\hline  
			0xf0 & CREATE & 32000 & 3 & 1 & Create a new contract account. Operand order is: value, input offset, input size. \\
			0xf1 & CALL & 700 & 7 & 1 & Message-call into an account. The operand order is: gas, to, value, in offset, in size, out offset, out size. \\
			0xf2 & CALLCODE & 700 & 7 & 1 & Message-call into this account with an alternative account's code. Exactly equivalent to CALL, except the recipient is the same account as at present, but the code is overwritten. \\
			0xf3 & RETURN & 0 & 2 & 0 & Halt execution, then return output data. This defines the output at the moment of the halt. \\
			0xf4 & DELEGATECALL & 700 & 6 & 1 & Message-call into this account with an alternative account's code, but with persisting values for \texttt{sender} and \texttt{value}. DELEGATECALL takes one less argument than CALL. This means that the recipient is in fact the same account as at present, but that the code is overwritten \textit{and} the context is almost entirely identical. \\
			0xf5 & CALLBLACKBOX & 40 & 7 & 1 & - \\
			0xfa & STATICCALL & 40 & 6 & 1 & - \\
			0xfd & REVERT & 0 & 2 & 0 & - \\
			0xfe & INVALID & - & 1 & 0 & Designated invalid instruction. \\
			0xff & SELFDESTRUCT & 5000 & 1 & 0 & Halt execution and register the account for later deletion. \\
			\hline
			\end{longtable}


\printbibliography
\end{document}


