\documentclass[10pt,a4paper,oneside]{scrartcl}
\usepackage[latin1]{inputenc}
\usepackage{amsmath}
\usepackage{amsfonts}
\usepackage{amssymb}
\usepackage{makeidx}
\usepackage{graphicx}
\usepackage{booktabs}
\usepackage[
	authordate,
	strict,
	backend=biber
]
{biblatex-chicago}
\usepackage{mathtools}
\author{}
\title{Bit}
\date{}
\addbibresource{~/modules/References.bib}
\begin{document}
\maketitle
\paragraph{Notation}: \texttt{bit}
\paragraph{Description}: The smallest unit of electronic data storage: there are eight bits in one byte. The Yellowpaper makes certain values appear bigger by giving them in bits (e.g. 160 bits instead of 20 bytes.)
\end{document}

