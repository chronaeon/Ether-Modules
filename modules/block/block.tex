\documentclass[10pt,a4paper,oneside]{scrartcl}
\usepackage[latin1]{inputenc}
\usepackage{amsmath}
\usepackage{amsfonts}
\usepackage{amssymb}
\usepackage{makeidx}
\usepackage{graphicx}
\usepackage{booktabs}
\usepackage[
	authordate,
	strict,
	backend=biber
]
{biblatex-chicago}
\usepackage{mathtools}
\author{}
\title{Block}
\date{}
\addbibresource{~/modules/References.bib}
\begin{document}
\maketitle
\paragraph{Notation}: $\mathbb{B}$
\paragraph{Description}: A block is made up of 17 different elements, all of which play a unique role in the Blockchain. The first 15 elements that make up the block are part of what is called the \textsl{block header}. 

  \subsubsection{The Block}
                                A block is made up of 17 different elements. The first 15 elements are part of what is called the \textsl{block header}.


                                \subsubsection{Block Header}
                                \paragraph{Notation}: \texttt{header}
                                \paragraph{Description}: The information contained in a block besides the transactions list. This consists of:

                                \begin{enumerate}
                                        \item \textbf{Parent Hash} -- This is the Keccak-256 hash of the parent block's header.
                                        \item \textbf{Ommers Hash} -- This is the Keccak-256 hash of the ommer's list portion of this block.
                                        \item \textbf{Beneficiary} -- This is the 20-byte address to which all block rewards are transferred.
                                        \item \textbf{State Root} -- This is the Keccak-256 hash of the root node of the state trie, after a block and its transactions are finalized.
                                        \item \textbf{Transactions Root} -- This is the Keccak-256 hash of the root node of the trie structure populated with each transaction from a Block's transaction list.
                                        \item \textbf{Receipts Root} -- This is the Keccak-256 hash of the root node of the trie structure populated with the receipts of each transaction in the transactions list portion of the block.
                                        \item \textbf{Logs Bloom} -- This is the bloom filter composed from indexable information (log address and log topic) contained in the receipt for each transaction in the transactions list portion of a block.
                                        \item \textbf{Difficulty} -- This is the difficulty of this block -- a quantity calculated from the previous block's difficulty and its timestamp.
                                        \item \textbf{Number} -- This is a quantity equal to the number of ancestor blocks behind the current block.
                                        \item \textbf{Gas Limit} -- This is a quantity equal to the current maximum gas expenditure per block.
                                        \item \textbf{Gas Used} -- This is a quantity equal to the total gas used in transactions in this block.
                                        \item \textbf{Timestamp} -- This is a record of Unix's time at this block's inception.
                                        \item \textbf{Extra Data} -- This byte-array of size 32 bytes or less contains extra data relevant to this block.
                                        \item \textbf{Mix Hash} -- This is a 32-byte hash that verifies a sufficient amount of computation has been done on this block.
                                        \item \textbf{Nonce} -- This is an 8-byte hash that verifies a sufficient amount of computation has been done on this block.
                                        \item \textbf{Ommer Block Headers} -- These are the same components listed above for any ommers.

                                        \subsubsection{Block Footer}
                                        \item \textbf{Transaction Series} -- This is the only non-header content in the block.
                                \end{enumerate}
                                                                                                                                     278,1-8       35%


\end{document}

