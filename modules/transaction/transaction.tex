\documentclass[10pt,a4paper,oneside]{scrartcl}
\usepackage[latin1]{inputenc}
\usepackage{amsmath}
\usepackage{amsfonts}
\usepackage{amssymb}
\usepackage{makeidx}
\usepackage{graphicx}
\usepackage{booktabs}
\usepackage[
	authordate,
	strict,
	backend=biber
]
{biblatex-chicago}
\usepackage{mathtools}
\author{}
\title{Transaction}
\date{}
\addbibresource{~/modules/References.bib}
\begin{document}
\maketitle
\paragraph{Notation}: \texttt{transaction}
\paragraph{Description}: A single cryptographically signed instruction sent to the Ethereum network. There are two types of transactions: \textsc{message calls} and \textsc{contract creation}. Transactions are ubiquitous on the Ethereum network, and represent several common fields. 

\begin{enumerate}

	\item \textbf{Nonce} -- The number of transactions sent by the sender.
	\item \textbf{Gas Price} -- The number of Wei to pay the network for unit of gas.
	\item \textbf{Gas Limit} -- The maximum amount of gas to be used in while executing a transaction. 

	\item \textbf{To} -- The 20-character recipient of a message call.\footnote{In the case of a contract creation this is 0x000000000000000000.}
	\item \textbf{Value} The number of Wei to be transferred to the recipient of a message call.\footnote{In the case of a contract creation, an endowment to the newly created contract account.}
	\item \textbf{v, r, s} 
	\item \textbf{}
	\item 
\end{enumerate}

\end{document}

