\documentclass[10pt,a4paper,oneside]{scrartcl}
\usepackage[latin1]{inputenc}
\usepackage{amsmath}
\usepackage{amsfonts}
\usepackage{amssymb}
\usepackage{makeidx}
\usepackage{graphicx}
\usepackage{booktabs}
\usepackage[
	authordate,
	strict,
	backend=biber
]
{biblatex-chicago}
\usepackage{mathtools}
\author{}
\title{Block}
\date{}
\addbibresource{~/modules/References.bib}
\begin{document}
\maketitle
\paragraph{Notation}: $\mathbb{B}$
\paragraph{Description}: A block is made up of 17 different elements, all of which play a unique role in the Blockchain. The first 15 elements that make up the block are part of what is called the \textsl{block header}. 

\begin{enumerate}
	\item \textbf{Parent Hash}: The hash of the parent block's header: given in Keccak-256.
	\item \textbf{Ommers Hash}: The hash of the \textsl{Ommers List} portion of this block: given in Keccak-256.
	\item \textbf{Beneficiary}: The 20-character (160-bit) hash to which block rewards are transferred when the block is successfully mined.
	\item \textbf{State Root}: The Keccak-256 bit hash of the root node of the state. 


\end{enumerate}

\end{document}

