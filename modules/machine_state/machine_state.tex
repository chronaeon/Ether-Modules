\documentclass[10pt,a4paper,oneside]{scrartcl}
\usepackage[latin1]{inputenc}
\usepackage{amsmath}
\usepackage{amsfonts}
\usepackage{amssymb}
\usepackage{makeidx}
\usepackage{graphicx}
\usepackage{booktabs}
\usepackage[
	   style=ieee
	   ]
	   {biblatex}
\usepackage{mathtools}
\author{}
\title{Machine State}
\date{}
\addbibresource{~/modules/References.bib}
\begin{document}
\maketitle
\paragraph{Notation}: \texttt{machine\_state}
\paragraph{Description}: The machine state is a tuple consisting of five elements:

	\begin{enumerate}
		\item \texttt{gas\_available}
		\item \texttt{program\_counter}
		\item \texttt{memory\_contents} A series of zeroes of size $2^{256}$
		\item \texttt{memory\_words.count}
		\item \texttt{stack\_contents}
	\end{enumerate}
	
	There is also, \texttt[to\_execute]: the current operation to be executed

\subsubsection{Exceptional Halting}
An exceptional halt may be caused by a handful of boolean values:

\begin{verbatim}
	forall instruction.x
	if gas_empty = true
	then signal halt
	elif instruction.x = fake
	then signal halt
	elif stack = terse
	then signal halt
	elif jumpdest = bad
	then signal halt
	else exec instruction.x
	
	forall instruction.y 
	[...]
	[...]
	[...]
	[...]

	forall instruction.z 
	[...]
	[...]
	[...]
	[...]

	then signal controlled_halt
\end{verbatim}

No instruction can, through its execution, cause an exceptional halt. They can only happen if some instruction, for whatever reason, fails to execute.

\begin{itemize}
	\item The amount of remaining gas in each transaction is extracted from information contained in the \texttt{machine\_state} 
	\item A simple iterative recursive  loop\supercite{Wood2017} with a boolean  value: 
		\begin{itemize}
		\item	[true] indicating that in the run of computation, an exception was signaled
		\item	[false] indicating in the run of computation, exceptions were signaled. If this value remains false for the duration of the execution until the set of transactions becomes a series (rather than an empty set.) This means that the machine has reached a controlled halt. 
		\end{itemize}
\end{itemize}


\printbibliography
\end{document}

