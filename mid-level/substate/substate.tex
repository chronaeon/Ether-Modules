\documentclass[10pt,a4paper,oneside]{scrartcl}
\usepackage[latin1]{inputenc}
\usepackage{amsmath}
\usepackage{amsfonts}
\usepackage{amssymb}
\usepackage{makeidx}
\usepackage{graphicx}
\usepackage{booktabs}
\usepackage[
	authordate,
	strict,
	backend=biber
]
{biblatex-chicago}
\usepackage{mathtools}
\author{}
\title{Substate}
\date{}
\addbibresource{~/modules/References.bib}
\begin{document}
\maketitle
\paragraph{Notation}: \texttt{substate}
\paragraph{Description}: A smaller, temporary state that is generated during transaction execution. It contains three sets of data:

\begin{itemize}
        \item The accounts tagged for self-destruction following the transaction's completion. \texttt{self\_destruct(accounts)}
        \item The \texttt{logs\_series}, which creates checkpoints in EVM code execution for frontend applications to explore, and is made up of the\texttt{logs\_set} and \texttt{l
ogs\_bloom} from the \texttt{tx\_receipt}.
	\item The refund balance.\footnote{The \textsc{sstore} operation increases the amount refunded by resetting contract storage to zero from some non-zero state.}
\end{itemize}
 

\end{document}

