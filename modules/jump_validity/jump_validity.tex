\documentclass[9pt,a4paper,oneside]{scrartcl}
\usepackage[latin1]{inputenc}
\usepackage{amsmath}
\usepackage{amsfonts}
\usepackage{amssymb}
\usepackage{makeidx}
\usepackage{graphicx}
\usepackage{booktabs}
\usepackage[
	   style=ieee
	   ]
	   {biblatex}
\usepackage[a4paper,width=170mm,top=18mm,bottom=22mm,includeheadfoot]{geometry}
\usepackage{mathtools}
\usepackage{concmath}
\usepackage[T1]{fontenc}
\author{}
\title{Jumpdest Validity}
\date{}
\addbibresource{~/modules/References.bib}
\begin{document}
\maketitle
\paragraph{Notation}: \texttt{function jumpdest\_valid}
\paragraph{Description}: This function determines the \textit{set} of valid jump destinations given the code being run. This counts toward any position in the EVM code occupied by a \textsc{jumpdest} instruction. All of such positions in EVM code need to be on valid instruction boundaries, that is, not in the data portion of a \textsc{push} operation, and not in any trailing \textsc{push} operations. 

Formally:
\begin{equation}
	D(\mathbf{c}) \equiv D_J(\mathbf{c}, 0)
\end{equation}

where:
\begin{equation}
	D_J(\mathbf{c}, i) \equiv \begin{cases}
		\{\} & \text{if} \quad i \geqslant |\mathbf{c}|  \\
		\{ i \} \cup D_J(\mathbf{c}, N(i, \mathbf{c}[i])) & \text{if} \quad \mathbf{c}[i] = \text{\small JUMPDEST} \\
		D_J(\mathbf{c}, N(i, \mathbf{c}[i])) & \text{otherwise} \\
	\end{cases}
\end{equation}

where $N$ is the next valid instruction position in the code, skipping the data of a {\small PUSH} instruction, if any:
\begin{equation}
	N(i, w) \equiv \begin{cases}
		i + w - \text{\small PUSH1} + 2 & \text{if} \quad w \in [\text{\small PUSH1}, \text{\small PUSH32}] \\
	i + 1 & \text{otherwise} \end{cases}
\end{equation}




\printbibliography
\end{document}

