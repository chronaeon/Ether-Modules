\documentclass[10pt,a4paper,oneside]{scrartcl}
\usepackage[latin1]{inputenc}
\usepackage{amsmath}
\usepackage{amsfonts}
\usepackage{amssymb}
\usepackage{makeidx}
\usepackage{graphicx}
\usepackage{booktabs}
\usepackage[
	authordate,
	strict,
	backend=biber
]
{biblatex-chicago}
\usepackage{mathtools}
\author{}
\title{Balance}
\date{}
\addbibresource{~/modules/References.bib}
\begin{document}
\maketitle
\paragraph{Notation}: \texttt{bal}
\paragraph{Description}: The amount of \textbf{Wei} \textsc{owned} by this account. \footnote{This formal notation is $\sigma[a]_b$}
    \begin{itemize}
        \item Key/value pair stored inside the root hash. \footnote{$\texttt{\small TRIE}\big(L_I^*(\boldsymbol{\sigma}[a]_\mathbf{s})\big) \equiv \boldsymbol{\sigma}[a]_s$}
        \item $L_I^*$, is defined as the element-wise transformation of the base function\footnote{$L_I$, given as: $L_I\big( (k, v) \big) \equiv \big(\texttt{\small KEC}(k
), \texttt{\small RLP}(v)\big)$}
        \item The \textsl{element-wise transformation of the base-function} refers to all of the key/value pairs in \textit{$L_I$}
        \item $L_I$ refers to a particular \gls{trie}.
    \end{itemize}
\end{document}

