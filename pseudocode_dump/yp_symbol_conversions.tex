\documentclass[9pt,a4paper,oneside]{scrartcl}
\usepackage[latin1]{inputenc}
\usepackage{amsmath}
\usepackage{amsfonts}
\usepackage{amssymb}
\usepackage{makeidx}
\usepackage{graphicx}
\usepackage{booktabs}
\usepackage[style=ieee]{biblatex}
\usepackage[a4paper,width=170mm,top=18mm,bottom=22mm,includeheadfoot]{geometry}
\usepackage{mathtools}
\usepackage{concmath}
\usepackage[T1]{fontenc}
\author{}
\title{Yellowpaper Pseudocode Equivalents}
\date{}
\addbibresource{~/modules/References.bib}
\begin{document}
\maketitle
\centering
\begin{tabular}{rl}
	\hline
	\textbf{YP Notation} & \textbf{Pseudocode} \\
	\hline
	$\boldsymbol{\sigma}$ & \texttt{actual\_state} \\
	$\mu$ & \texttt{actualizing\_state} \\
	$\Upsilon$ & \texttt{state\_actualization\_function} \\
	$\Pi$ & \texttt{block\_level\_state\_actualization\_function} \\
	$\Omega$ & \texttt{block\_finalization\_state\_actualization\_function} \\
	$\mathbb{B}$ & \texttt{all\_byte\_sequences(set)} \\
	$\mathbb{P}$ & \texttt{all\_non\_negative\_scalars(set)} \\
	\textit{B} & \texttt{actualizing\_block} \\
	\textit{C} & \texttt{operation\_cost} \\
	$\mathcal{L}$ & \texttt{last\_item\_in(set)} \\
	$\delta$ & \texttt{stack\_inputs} \\
	$\equiv$ & $=$ \\

	\hline
\end{tabular}


\printbibliography
\end{document}

