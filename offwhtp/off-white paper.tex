\documentclass[10pt,a4paper,leqno,bibliography=totoc]{scrartcl}
\usepackage[utf8]{inputenc}
\usepackage{lastpage} % Required to print the total number of pages
\usepackage[left=1.3cm,right=1.3cm,top=1.8cm,bottom=4.0cm]{geometry} % Adjust page margins
\usepackage{amsmath} % Required for equation customization
\usepackage{amssymb} % Required to include mathematical symbols
\usepackage{makecell}
\usepackage{boldline}
\usepackage{beton}
\usepackage{booktabs}
\usepackage[T1]{fontenc}
\usepackage[normalem]{ulem}
\usepackage{array}
\usepackage[colorlinks]{hyperref}
\usepackage{graphicx}
\usepackage{multicol}
\usepackage{setspace}
\usepackage[title,titletoc]{appendix}
\usepackage{mathtools}
\usepackage{subfig}
\usepackage{ragged2e}
\usepackage[singlelinecheck=false]{caption}
\usepackage{wrapfig}
\usepackage{IEEEtrantools}
\usepackage{lipsum}
\usepackage[english]{babel}
\usepackage{blindtext}
\usepackage{pgf}
\usepackage{tikz}
\usetikzlibrary{mindmap,arrows,automata,shadows}
\usepackage{epigraph}
\usepackage{tabularx}
\usepackage[perpage]{footmisc}
\usepackage[style=ieee,
	backref=true
	]{biblatex}
\usepackage{etoolbox}
\usepackage{framed,color}
\usepackage[framemethod=tikz]{mdframed}
\usepackage{minitoc}
\usepackage{fancybox}
\usepackage{tablefootnote}
\usepackage{datatool}
\usepackage{footnote}
\usepackage{pgfkeys}
\usepackage{pgf}
\usepackage{microtype}
\usepackage{enumitem}
\usepackage[acronyms,toc,section=section]{glossaries}
\usepackage{endnotes}
\usepackage{dirtytalk}
\usepackage{pdfpages}
\usepackage{supertabular}
\usepackage{kantlipsum}
\usepackage{makeidx}
\makeindex
\usepackage[totoc]{idxlayout}
\usepackage{longtable}

\glstoctrue
\newmdenv[shadow=true,shadowcolor=black,font=\sffamily,rightmargin=8pt]{shadedbox}

%\definecolor{papayawhip}{RGB}{255,239,213}
\definecolor{blanchedalmond}{RGB}{255,235,205}

\newcount\n
\n=0
\def\tablebody{}
\makeatletter
\loop\ifnum\n<100
 \advance\n by1
 \protected@edef\tablebody{\tablebody
 \textbf{\number\n.}& shortText
 \tabularnewline
  }
\repeat
\makeatletter
			\let\mcnewpage=\newpage
		\newcommand{\TrickSupertabularIntoMulticols}{%
\renewcommand\newpage{%
	      \if@firstcolumn
	            \hrule width\linewidth height0pt
          \columnbreak
      \else
        \mcnewpage
       \fi
  }%
 }
 \makeatother


\makenoidxglossaries
\setacronymstyle{long-short}
\newacronym{OOP}{OOP}{Object-Oriented Programming}
\newacronym{EVM}{EVM}{Ethereum Virtual Machine}
\newacronym{YP}{YP}{Yellowpaper}
\newacronym{ERE}{ERE}{Ethereum Runtime Environment}
\newacronym{LLL}{LLL}{Lower Level Lisp}
\newglossaryentry{serialization}{name={serialization}, description={Serialization is the process of converting an object into a stream of bytes in order to store the object or transmit it to memory, a database, or a file. Its main purpose is to save the state of an object in order to be able to recreate it when needed. The reverse process is called deserialization.\autocite{billwagner}}}
\newglossaryentry{state database}{name={state database},description={A database stored off-chain, [i.e. on the computer of some user running an Ethereum client] which contains a trie structure mapping bytearrays [i.e. organized chunks of binary data] to other bytearrays [other organized chunks of binary data]. The \textsc{relationships} between each node on this trie constitute a \textsc{map}, a.k.a. a \textsc{mapping} of all \textsc{previous states} on the \textbf{EVM} which a client might need to reference}}
\newglossaryentry{abstract machine}{name={abstract machine}, description={An abstract machine is a conceptual model of a computer that describes its own operations with perfect accuracy. Since abstract machines are theoretical, all possible outputs can be determined beforehand}}
\newglossaryentry{design pattern}{name={design pattern}, description={a pattern of design in OOP}}
\newglossaryentry{transaction}{name={transaction}, description={An input message to a system that, because of the nature of the real-world event or activity it reflects, is required to be regarded as a single unit of work guaranteeing to either be processed completely or not at all.\autocite{Ngondi2016}}} 
\newglossaryentry{hacker ethic}{name={hacker ethic}, description={A maxim purporting that knowledge about and access to proprietary computer systems should be free and unhindered for anyone who is willing and able to explore and improve it.\autocite{Levy2010} The open-source and decentralized nature of Ethereum makes it one of the most thorough and robust implementations of the \textsl{Hacker Ethic} to date.}}
\newglossaryentry{Ethereum Virtual Machine}{name={Ethereum Virtual Machine}, description={A sub-process of the \textit{State Transition Function} which initializes and executes all of the transactions (ergo computations) in a block, prior to their finalization into the state.}}
\newglossaryentry{blockchain}{name={blockchain}, description={A consensus-based record of agreement where chunks of data\footnote{As of \today, roughly between 1 and 25 kilobytes in size\autocite{Etherscan2017}} (called blocks) are stored with cryptographic hashes linking each Block to the next, ensuring their validity. See also: \textit{hashing functions}.}}
\newglossaryentry{Cryptographic hashing functions}{name={Cryptographic hashing functions}, description={Cryptographic hashing is what makes Blockchain technology possible. We take for granted that from a certain hash function, (say SHA-256 in the case of Bitcoin) there are a certain, limited number of hashes from the previous block's nonce function as solutions to an equation. Out there, somewhere in the world of numbers, there already exist cryptographic hashes that fit the requirements of the current block's nonce. Since it's impossible for someone to convincingly fake the current block, due to everyone running an identical protocol, bad blocks are spotted. Good solutions to the next true block are attempted instead, since it's easier (though still notably difficult)  to mine for the next hash, that is, to search for the next needle-in-a-haystack that solves the equation, than it is to try and fool the peer-to-peer network into agreeing on a false block. A classic game of economics is here at play, and it runs assuming man's inherent self-interest. Not the bad kind of self-interest, but the good kind that is interested in a self's healthy growth and development. If there were not these hashing functions for us to utilize from broader studies in Mathematics, the entire cryptocurrency market and concept would fall like a house of cards. The fact that these networks are still up and running proves that the Math works. The biggest problem for hashing is the emergence of hash collisions, where two inputs produce the same output hash. These sometimes take years to find, and with a good enough hashing function usually are not found until the next more powerful hashing function has arrived on the scene.}}
\newglossaryentry{state machine}{name={state machine}, description={A state machine rests in a universal, stable, singular condition, called a state. State machines  transition to new states given certain compatible inputs.}}
\newglossaryentry{Ethereum Foundation}{name={Ethereum Foundation}, description={The non-profit organization in charge of executing the development processes of Ethereum in line with the \gls{Whitepaper}}}
\newglossaryentry{Whitepaper}{name={Whitepaper}, description={A conceptual map, distinct from the Yellowpaper, which highlights the development goals for Ethereum as a whole\autocite{EF2017}}}
\newglossaryentry{Yellowpaper}{name={Yellowpaper}, description={Ethereum's primary formal specification, written by Dr. Gavin Wood, one of the founders of Ethereum. }}
\newglossaryentry{singleton}{name={singleton}, description={A design pattern in Object-Oriented Programmdng which specifies a class with one instance but with a global point of access to it\autocite{sourcemaking}}}
\newglossaryentry{specification}{name={specification}, description={Technical descriptions, instructions, and definitions from which other people can create working prototypes}}
\newglossaryentry{State Database}{name={State Database}, description={A database backend that maintains a mapping of bytearrays to bytearrays}}
\newglossaryentry{root node}{name={root node}, description={the uppermost node in a particular tree, of blocks, representing a single world state$^\sigma$ at a particular time}}
\newglossaryentry{trie}{name={trie}, description={A tree-structure for organizing data, the position of data in the tree contains the particular path from root to leaf node that represents the key (the path from root to leaf is ``one'' key) you are searching the trie structure for. The data of the key is contained in the trie relationships that emerge from related nodes in the trie structure}}
\newglossaryentry{leaf node}{name={leaf node},description={the bottom-most node in a particular tree, of blocks, one half of the ``key'' the other half being the root node, which creates the path between}}
\newglossaryentry{addresses}{name={addresses}, description={160-bit (20-byte) identifiers--the last 20 characters of an Ethereum address}}
\newglossaryentry{public key}{name={public key}, description={A term originating from \textsl{cryptography} and corresponding to  \textbf{private key}, this is the 42-byte (i.e. 42-character) string of ASCII digits which transacts on the public network}}
\newglossaryentry{EVM Code}{name={EVM Code}, description={The bytecode that the EVM can natively execute. Used to formally specify the meaning and ramifications of a message to an Account}}
\newglossaryentry{EVM Assembly}{name={EVM Assembly}, description={The human readable version of EVM code}}
\newglossaryentry{External Actor}{name={External Actor}, description={A person or other entity able to interface to an Ethereum node, but external to the world of Ethereum. It can interact with Ethereum through depositing signed Transactions and inspecting the blockchain and associated state. Has one (or more) intrinsic Accounts}}
%\newglossaryentry{Intrinsic Account}{name={Instrinsic Account}, description={An}}
\newglossaryentry{Address}{name={Address}, description={A 160-bit (20-byte) code used for identifying Accounts}}
\newglossaryentry{Bit}{name={Bit}, description={The smallest unit of electronic data storage: there are eight bits in one byte. The Yellowpaper gives certain values in bits (e.g. 160 bits instead of 20 bytes)}}
\newglossaryentry{YP-transaction}{name={Transaction}, description={A piece of data, signed by an External Actor. It represents either a Message or a new Autonomous Object. Transactions are recorded into each block of the blockchain}}
\newglossaryentry{Autonomous Object}{name={Autonomous Object}, description={A notional object existent only within the hypothetical state of Ethereum. Has an intrinsic address and thus an associated account; the account will have non-empty associated EVM Code. Incorporated only as the Storage State of that account}}
\newglossaryentry{Storage State}{name={Storage State}, description={The information particular to a given Account that is maintained between the times that the Account's associated EVM Code runs}}
\newglossaryentry{Message}{name={Message}, description={Data (as a set of bytes) and Value (specified in Wei) that is passed between two Accounts, either through the deterministic operation of an Autonomous Object or the cryptographically secure signature of the Transaction}}
\newglossaryentry{Message Call}{name={Message}, description={The act of passing a message from one Account to another. If the destination account is associated with non-empty EVM Code, then the VM will be started with the state of said Object and the Message acted upon. If the message sender is an Autonomous Object, then the Call passes any data returned from the VM operation}}
\newglossaryentry{Gas}{name={Gas}, description={The fundamental network cost unit. Paid for exclusively by Ether (as of PoC-4), which is converted freely to and from Gas as required. Gas does not exist outside of the internal Ethereum computation engine; its price is set by the Transaction and miners are free to ignore Transactions whose Gas price is too low}}
\newglossaryentry{Contract}{name={Contract}, description={A piece of EVM Code that may be associated with an Account or an Autonomous Object}}
\newglossaryentry{Object}{name={Object}, description={Synonym for Autonomous Object}}
\newglossaryentry{Dapp}{name={Dapp}, description={An end-user-visible application hosted in an Ethereum browser.}}
\newglossaryentry{Ethereum Browser}{name={Ethereum Browser}, description={\footnote{a.k.a. Ethereum Reference Client} A cross-platform GUI of an interface similar to a simplified browser (a la Chrome) that is able to host applications, the backend of which is purely on the Ethereum protocol.}}
\newglossaryentry{Ethereum Runtime Environment}{name={Ethereum Runtime Environment}, description={The environment which is provided to an Autonomous Object executing in the EVM. Includes the EVM but also the structure of the world state on which the relies for certain I/O instructions including CALL \& CREATE}}
\newglossaryentry{Lower-Level Lisp}{name={Lower-Level Lisp}, description={The Lisp-like Low-level Language, a human-writable language used for authoring simple contracts and general low-level language toolkit for trans-compiling to}}
\newglossaryentry{Balance}{name={Balance}, description={A value which is intrinsic to accounts; the quantity of Wei in the account. All EVM operations are associated with changes in account balance}}
\newglossaryentry{beneficiary}{name={beneficiary}, description={The 160-bit address to which all fees collected from the successful mining of this block be transferred}}
\newglossaryentry{block header}{name={block header}, description={The information in a block besides transaction information. It consists of a dozen parts: (lists the 12 parts) }}
\newglossaryentry{state}{name={state}, description={A permanent, static, state  \texttt{state}}}
\newglossaryentry{transition}{name={state-transition}, description={}}


\addbibresource{References.bib}


%\makeatletter
%\patchcmd{\footnotetext}{\footnotesize}{\small\sffamily}{}{}
%\makeatother

\def\thesection{\arabic{section}}
%\renewcommand\thesubsection{\thesection\arabic{subsection}}


\setlength\bibitemsep{1.5\itemsep}
\newcommand{\sortitem}[1]{%
\DTLnewrow{list}% Create a new entry
\DTLnewdbentry{list}{description}{#1}% Add entry as description
}
\newenvironment{sortedlist}{%
\DTLifdbexists{list}{\DTLcleardb{list}}{\DTLnewdb{list}}% Create new/discard old list
}{%
\DTLsort{description}{list}% Sort list
\DTLforeach*{list}{\theDesc=description}{%
\item \theDesc}% Print each item

}


\newenvironment{alphafootnotes}
{\par\edef\savedfootnotenumber{\number\value{footnote}}
\renewcommand{\thefootnote}{\alph{footnote}}
\setcounter{footnote}{0}}
{\par\setcounter{footnote}{\savedfootnotenumber}}


\setlength{\columnseprule}{0pt}

\setlength{\columnsep}{5mm}

\hfuzz=0.84074pt

\setcounter{secnumdepth}{2}

\author{\textbf{M.D. Dameron}}

\title{\Huge{Off-White Paper} \\ \hfill \\ \Large{A Technical Description of Ethereum}}

\makeindex

\begin{document}

	\setstretch{1.15}

	\begin{alphafootnotes}

	\pagecolor{blanchedalmond}

	\maketitle

	\epigraph{\textsl{Beautiful is better than ugly. \\
	Explicit is better than implicit. \\
	Simple is better than complex. \\
	Complex is better than complicated.
	}}{\textit{The Zen of Python}}


\hfill \hfill
\begin{center}\textbf{Abstract}\end{center}\par
	\abstract{The goal of this paper\footnote{Formally, \textit{Blanched-Almond Paper}}} is to create and expand concepts from Ethereum around which, notwithstanding any earlier documentation, there may be some justified confusion. We use pseudocode to describe Ethereum's operation rather than formal math and greek symbols, because pseudocode is actually \textit{preferable} when describing \textsc{Abstract State Machines},\footnote{E. Borger and S. Robert F., \textit{Abstract state machines}:\textsl{ A method for high-level system design and analysis}. 1, pp. 3-8. Springer, 2003.} which Ethereum is. This paper takes an approach to describing Ethereum that focuses on clarity and approachability. Our prime source has been the Ethereum \textit{Yellowpaper}, but much supplemental knowledge has been found elsewhere and crucial points from other sources have been added as well for the reader's benefit.
	

\clearpage
%dominitoc 

\begin{center}\section*{Acknowledgements}\end{center}

 Thank you to the Ethereum founders for creating a product worth writing about in minute detail. Thanks to the Ethereum Foundation for maintaining this product in its basic integrity. Thanks to the ConsenSys Mesh for supporting my work on this project, by contributing your vast knowledge and expertise. Finally, thank you to Dr. Gavin Wood for your technical astuteness and creative genius; your Yellowpaper has given Ethereum a soul worth decoding.

\clearpage

\tableofcontents

\clearpage

\begin{multicols*}{2}
\TrickSupertabularIntoMulticols
\justify

\part{Theory}

	\section{The Evolution of a Protocol}
	Over the past decade, blockchain technology has proven its longevity and veracity through a number of systems, most notably through Bitcoin, the first electronic currency of its kind to succeed. Bitcoin was successful in its mission to create currency based on a decentralized peer-to-peer Blockchain protocol, and Ethereum takes that concept a step further by creating a \textit{globally-distributed virtual machine}that can ad-hoc run such currency applications, along with any other conceivable applications or programs. Using well-established concepts from the relevant areas of computer science, like \textsc{message-passing}, \textsc{transaction processing}, and \textsc{shared-state concurrence}, the \textsl{Ethereum Protocol} creates an environment for developers to execute machine instructions with the same level of veracity and certainty as monetary transactions have on more standard Blockchains.  
	\index{blockchain!veracity}\index{currency!electronic}\index{Bitcoin}\index{state}\index{message-passing}\index{shared-state!concurrence}	

\paragraph{A Conceptual Map of the Ethereum Protocol}

	\begin{savenotes}
		\begin{tikzpicture} [font=\scriptsize,
		grow cyclic, 
		level 1/.style={level distance=1.4cm,sibling angle=135},
		level 2/.style={text width=2cm, font=\footnotesize, level distance=2cm,sibling angle=45}]
		\node {Ethereum} % root
		child { node {Memory}
		child { node {Merkle-Patricia Trie} }
		child { node {Blockdrive} }
		child { node {State} }
		child { node {Hash Functions} }
		}
		child { node {CPU}
		child { node {State Transitions} }
		child { node {Opcodes} }
		child { node {EVM Code} }
		child { node {Propogation Time} }
		}
		child { node {Theory}
		child { node {Cryptoeconomics} }
		child { node {Game Theory} }
		child { node {Organizational Design} }
		};
		\end{tikzpicture}
	\end{savenotes}
	\section{Data Structures}
	\section{Cryptography} Cryptographic hashing is what makes Blockchain technology possible. We take for granted that from a certain hash function, (say SHA-256 in the case of Bitcoin) there are a certain, limited number of hashes from the previous block's nonce function as solutions to an equation. Out there, somewhere in the world of numbers, there already exist cryptographic hashes that fit the requirements of the current block's nonce. Since it's impossible for someone to convincingly fake the current block, due to everyone running an identical protocol, bad blocks are spotted. Good solutions to the next true block are attempted instead, since it's easier (though still notably difficult)  to mine for the next hash, that is, to search for the next needle-in-a-haystack that solves the equation, than it is to try and fool the peer-to-peer network into agreeing on a false block. A classic game of economics is here at play, and it runs assuming man's inherent self-interest. Not the bad kind of self-interest, but the good kind that is interested in a self's healthy growth and development. If there were not these hashing functions for us to utilize from broader studies in Mathematics, the entire cryptocurrency market and concept would fall like a house of cards. The fact that these networks are still up and running proves that the Math works. The biggest problem for hashing is the emergence of hash collisions, where two inputs produce the same output hash. These sometimes take years to find, and with a good enough hashing function usually are not found until the next more powerful hashing function has arrived on the scene.
	\section{Serialization}
	\section{Hacker Ethic}
	\section{Message-Passing Interface}
	\section{Singleton Pattern}
	\section{State Machines}
	\section{Processor Technology}
	\section{Turing Machines}
		\subsection{Turing-Completeness}
	\section{Intrinsic Currency}
		The smallest unit of currency in Ethereum is the Wei, which is $1*10^(18)$ Ether. All units at the machine level are counted in Wei.
	\section{Computation Flow}

	\section{ECDSA}

		\subsection{Public Key Cryptography}
				\paragraph{Diffie Helman} hypothesis--stated in their paper (citation to it)
			\subsubsection{Public Keys}

			\subsubsection{Private Keys}

	\section{Programming Languages}

		\subsection{Lower-Level Lisp}
			and this is what turns your LLL code into EVM code and how it's executed in the evm
	
		\subsection{Solidity}
			and this is what turns your solidity code into EVM code and how it's executed in the EVM

\clearpage
\part{Practice}

	\section{Memory and Storage}
			
		\subsection{Data Structures}

			\subsection{Trees}
				\paragraph{Merkletrees}
				\paragraph{Merkle-Patricia Trees}

%-------			
				\subsubsection{World State}
					Also known simply as "state", this is a \textsc{mapping} of \textbf{addresses} and \textbf{account states} (RLP data structures), this is also known as \textit{state}, or $\sigma$. This mapping is not stored on the blockchain, rather it is stored as a Merkle-Patricia \gls{trie} in a \textsc{database backend}\footnote{A database backend is accessed by users indirectly through an external application, most likely an Ethereum client; see also: \gls{state database}} that maintains a mapping of bytearrays to bytearrays.\footnote{A bytearray is specific set of bytes [data] that can be loaded into memory. It is a structure for storing binary data, e.g. the contents of a file.} The cryptographic internal data going back to the \gls{root node} represents the \textit{State} of the Blockchain at any given root, i.e. at any given \textit{time}.\footnote{This permanent data structure makes it possible to easily recall any previous state with its root hash keeping the resources off-chain and minimizing on-chain storage needs.}As a whole, the state is the sum total of database relationships in the \textbf{ \gls{state database}}. The state is an inert position on the chain, a position between prior state and post state; a block's frame of reference, and a defined set of relationships to that frame of reference.
%--------


		\subsection{Byte Arrays}

		\subsection{Bit Sequences}
			Message Calls are either bit sequences or byte arrays.

		\subsection{The Block}

		\subsection{State Database}

		\subsection{Merkle-Patricia Trees}

			\subsubsection{RLP}
				Well-Formed RLP
	
			\subsubsection{Account State}
				The account state is the state of any particular account during some specified world state $\sigma$.\footnote{Formally \textbf{world state} $=$ $\sigma$.} \par

    				\paragraph{Nonce} 
					The \textbf{nonce} aspect of an \textsc{account's state} is the number of transactions sent from, or the number of contract-creations by, the address of that account.\footnote{$\sigma$ is the world state at a certain given time, and \textit{n} is the number of transactions or contract creations by that account.}
    				\paragraph{Storage Root}
    					The \textbf{storage root} aspect of an \textsc{account's state} is the hash of the trie\footnote{A particular path from root to leaf in the \textbf{\gls{state database}} that encodes the \textsc{storage contents} of the account.}
    				\paragraph{Code Hash}
    					The \textbf{code hash} aspect of an \textsc{account's state} is the \textsc{hash of the evm code} of this account. Code hashes are \textsc{stored} in the \textbf{\gls{state database}}. Code hashes are permanent and they are executed when the address belonging to that account \textsc{receives} a message call.\footnote{A message call is any interaction with the account on-chain.}\footnote{Formal notation is $\sigma[a]_c$}\footnote{More formal notation is $\texttt{\small KEC}(\mathbf{b}) = \boldsymbol{\sigma}[a]_c$}
    				\paragraph{Balance}
    					The amount of \textbf{Wei} \textsc{owned} by this account. \footnote{This formal notation is $\sigma[a]_b$}

		\subsection{Transaction Receipts}

   			 \subsubsection{Bloom Filter}
\clearpage

	\section{Processing and Computation}
	
		\subsection{State Transition Function}
		State Transitions come about through a what is known as the State Transition Function; this is an abstraction of several operations in Ethereum which comprise the overall act of computing changes to the \textit{machine state} prior to adding them to the \textit{world state}, that is, through them being finalized and rewards applied to a given miner. \texttt{apply\_rewards} and \texttt{block\_beneficiary} are here.

    		\subsection{Mining}

			\subsubsection{Ethash}
	
			\paragraph{GHOST Protocol}

		\subsection{Verification}
		    Verifies Ommer headers

			\subsubsection{Ommers}
				\paragraph{Ommershash}

			\subsubsection{\texttt{Is\_Sibling} Property}

		\subsection{Transactions}
			Transactions are the bread and butter of state transitions, that is of block additions, which are all the computation performed in one block. Each transaction applies the execution changes to the \textit{machine state}, a temporary state which consists of all the temporary changes in computation that must be made before a block is finalized and added to the world state.

		\subsection{Execution}

	
				\paragraph{Execution Model}
	
			\subsubsection{Message Calls}

			\subsubsection{Contract Creation}
		
			\subsubsection{Account Creation}
			
			\subsection{Halting}
				\paragraph{Execution Environment}
	
			\subsection{Gas}
	
				\paragraph{Miner Choice}
					Miners choose which gas prices they want to accept.
	
				\paragraph{Gasprice}

				\paragraph{Gaslimit}
					Any unused gas is refunded to the user.

				\paragraph{Gasused}
		
			\subsubsection{Machine State}

				\paragraph{Substate}
					The \texttt{substate} is an emergent, ever-changing ball of computational energy that is about to be applied to the main state. It is the \textit{meta state} by which transactions are decided valid and to be added to the blockchain.
		
			\subsubsection{EVM Code}

			\subsubsection{Opcodes}
				But what exactly are these computer instructions that can be executed with the same level of veracity and certainty as Bitcoin transactions? How do they come about, what makes them up, how are they kept in order, and what makes them execute? The first part of answering this question is understanding opcodes. In traditional machine architectures, you may not be introduced to working with processor-level assembly instructions for some time. In Ethereum however, they are essential to understanding the protocol because they are the most minute and subtle (yet HUGELY important) things going on in the Ethereum Blockchain at any moment, and they are the real "currency," that Ethereum trades in. I'll explain what I mean by that in a minute. First, let's go over a few Opcodes:\footnote{A full list of Opcodes is in Appendix B} \\

\begin{tabular}{|rclll|}
	\hline
	\textbf{Data} & \textbf{Opcode} & \textbf{Gas} & \textbf{In} & \textbf{Out}  \\
	\hline
	0x00 & STOP & 0 & 2 & 1 \\
	0x01 & ADD & 3 & 2 & 1 \\
	0x02 & MUL & 5 & 2 & 1 \\
	0x03 & SUB & 3 & 2 & 1 \\
	0x04 & DIV & 5 & 2 & 1 \\ 
	\hline
\end{tabular}

				\paragraph{}The STOP Opcode is used in order to stop a computation once it has completed, or to halt a computation if it has run out of gas. The ADD, MUL, SUB, and DIV operations are addition, multiplication, subtraction and division operations. The In/Out columns refer to inputs (to \texttt{machine\_state}), the state which decides every new \texttt{world\_state}. 

\clearpage
\end{multicols*}
\part{Appendix}

\appendix

\section{Opcodes}
\texttt{%
\begin{longtable}{|ccccc|}
\hline
\textbf{Data} & \textbf{Opcode} & \textbf{Gas}  & \textbf{Input}  & \textbf{Output}  \\
\hline
0x00 & STOP & 0 & 0 & 0 \\
0x01 & ADD & 3 & 2 & 1 \\
0x02 & MUL & 5 & 2 & 1 \\
0x03 & SUB & 3 & 2 & 1 \\
0x04 & DIV & 5 & 2 & 1 \\
0x05 & SDIV & 5 & 2 & 1 \\
0x06 & MOD & 5 & 2 & 1 \\
0x07 & SMOD & 5 & 2 & 1 \\
0x08 & ADDMOD & 8 & 3 & 1 \\
0x09 & MULMOD & 8 & 3 & 1 \\
0x0a & EXP & 10 & 2 & 1 \\
0x0b & SIGNEXTEND & 5  & 2 & 1 \\
0x10 & LT & 3 & 2 & 1 \\
0x11 & GT & 3 & 2 & 1 \\
0x12 & SLT & 3 & 2 & 1 \\
0x13 & SGT & 3 & 2 & 1 \\
0x14 & EQ & 3 & 2 & 1 \\
0x15 & ISZERO & 3 & 1 & 1 \\
0x16 & AND & 3 & 2 & 1 \\
0x17 & OR & 3 & 2 & 1 \\
0x18 & XOR & 3 & 2 & 1 \\
0x19 & NOT & 3 & 1 & 1 \\
0x1a & BYTE & 3 & 2 & 1 \\
0x20 & SHA3 & 30 & 2 & 1 \\
0x30 & ADDRESS & 2 & 0 & 1 \\
0x31 & BALANCE & 400 & 1 & 1 \\
0x32 & ORIGIN & 2 & 0 & 1 \\
0x33 & CALLER & 2 & 0 & 1 \\
0x34 & CALLVALUE & 2 & 0 & 1 \\
0x35 & CALLDATALOAD & 3 & 1 & 1 \\
0x36 & CALLDATASIZE & 2 & 0 & 1 \\
0x37 & CALLDATACOPY & 3 & 3 & 0 \\
0x38 & CODESIZE & 2 & 0 & 1 \\
0x39 & CODECOPY & 3 & 3 & 0 \\
0x3a & GASPRICE & 2 & 0 & 1 \\
0x3b & EXTCODESIZE & 700 & 1 & 1 \\
0x3c & EXTCODECOPY & 700 & 4 & 0 \\
0x3d & RETURNDATASIZE & 2 & 0 & 1 \\
0x3e & RETURNDATACOPY & 3 & 3 & 0 \\
0x40 & BLOCKHASH & 20 & 1 & 1 \\
0x41 & COINBASE & 2 & 0 & 1 \\
0x42 & TIMESTAMP & 2 & 0 & 1 \\
0x43 & NUMBER & 2  & 0 & 1 \\
0x44 & DIFFICULTY & 2 & 0 & 1 \\
0x45 & GASLIMIT & 2  & 0 & 1 \\
0x50 & POP & 2 & 1 & 0 \\
0x51 & MLOAD & 3 & 1 & 1 \\
0x52 & MSTORE & 3 & 2 & 0 \\
0x53 & MSTORE8 & 3 & 2 & 0 \\
0x54 & SLOAD & 200 & 1 & 1 \\
0x55 & SSTORE & 0 & 2 & 0 \\
0x56 & JUMP & 8 & 1 & 0 \\
0x57 & JUMPI & 10 & 2 & 0 \\
0x58 & PC & 2 & 0 & 1 \\
0x59 & MSIZE & 2 & 0 & 1 \\
0x5a & GAS & 2 & 0 & 1 \\
0x5b & JUMPDEST & 1 & 0 & 0 \\
0xa0 & LOG0 & 375 & 2 & 0 \\
0xa1 & LOG1 & 750 & 3 & 0 \\
0xa2 & LOG2 & 1125 & 4 & 0 \\
0xa3 & LOG3 & 1500 & 5 & 0 \\
0xa4 & LOG4 & 1875 & 6 & 0 \\
0xf0 & CREATE & 32000 & 3 & 1 \\
0xf1 & CALL & 700 & 7 & 1 \\
0xf2 & CALLCODE & 700 & 7 & 1 \\
0xf3 & RETURN & 0 & 2 & 0 \\
0xf4 & DELEGATECALL & 700 & 6 & 1 \\
0xf5 & CALLBLACKBOX & 40 & 7 & 1 \\
0xfa & STATICCALL & 40 & 6 & 1 \\
0xfd & REVERT & 0 & 2 & 0 \\
0xff & SUICIDE & 5000 & 1 & 1 \\
\hline
\end{longtable}
}


\clearpage
\begin{multicols*}{2}
\printbibliography
\clearpage
\printglossary[type=\acronymtype]
\glsaddall
\printnoidxglossaries
\clearpage
\end{multicols*}
\end{alphafootnotes}

\printindex


\end{document}
